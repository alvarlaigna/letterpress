\documentclass[runningheads]{llncs}

\usepackage{ngerman}

\usepackage[utf8]{inputenc}
\usepackage[T1]{fontenc}
\usepackage{llncsdoc}

\usepackage{amsmath}
\usepackage{amssymb}
\usepackage{enumerate}

\usepackage{amssymb}
\usepackage{hyperref}
\setcounter{tocdepth}{3}
\usepackage{graphicx}
\usepackage{url}


% fuer die aktuelle Zeit
\usepackage{scrtime}
\usepackage{listings}
\usepackage{subfigure}
\usepackage{hyperref}
\urldef{\mailsa}\path|{alfred.hofmann, ursula.barth, ingrid.beyer, christine.guenther,|
\urldef{\mailsb}\path|frank.holzwarth, anna.kramer, erika.siebert-cole, lncs}@springer.com|
\newcommand{\keywords}[1]{\par\addvspace\baselineskip
\noindent\keywordname\enspace\ignorespaces#1}

\begin{document}

\mainmatter  % start of an individual contribution

% first the title is needed
\title{####title####}

% a short form should be given in case it is too long for the running head
% \titlerunning{Lecture Notes in Computer Science: Authors' Instructions}

% the name(s) of the author(s) follow(s) next
%
% NB: Chinese authors should write their first names(s) in front of
% their surnames. This ensures that the names appear correctly in
% the running heads and the author index.
%

\author{####author####}
%
% \authorrunning{Lecture Notes in Computer Science: Authors' Instructions}

% (feature abused for this document to repeat the title also on left hand pages)

% the affiliations are given next
\institute{Springer-Verlag, Computer Science Editorial,\\
Tiergartenstr. 17, 69121 Heidelberg, Germany\\
\mailsa\\
\mailsb\\
\url{http://www.springer.com/lncs}}


\toctitle{Lecture Notes in Computer Science}
\tocauthor{Authors' Instructions}
\maketitle


\begin{abstract}
####abstract####
\end{abstract}


####content####

\end{document}